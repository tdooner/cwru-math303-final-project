%%%%%%%%%%%%%%%%%%%%%%%%%%%%%%%%%%%%%%%%%%%%%%%%%%%%%%%%%%%%%%%
\section{Background}
%%%%%%%%%%%%%%%%%%%%%%%%%%%%%%%%%%%%%%%%%%%%%%%%%%%%%%%%%%%%%%%

Prime numbers are some of the most important, interesting, and ultimately frustrating numbers. Prime numbers are integers that have no factors other one and themselves. They are seemingly randomly distributed along the number line and defy all efforts to untangle their mystery. 
Since there is no simple way to generate primes, it is challenging to compute and verify large
prime numbers. Although large prime numbers have little practical usage today, finding the
largest prime number is a mark of pride in the mathematical community. 
Today, Mersenne Primes are \textit{by far} the largest prime numbers.  
On September 15, 2008 the Great Internet Mersenne Prime Search announced that they had found the largest known prime $2^{43,112,609}-1$, a number that is over 12,000,000 digits long \cite{gimps}.

These numbers also have interesting properties that have made them a popular area to study. There is no sign that Mersenne Primes will lose their dominance of the search for ever larger primes any time soon. The study of these numbers began some 400 years ago with a French monk named Marin Mersenne.

%%%%%%%%%%%%%%%%%%%%%%%%%%%%%%%%%%%%%%%%%%%%%%%%%%%%%%%%%%%%%%%
\subsection{Marin Mersenne}
%%%%%%%%%%%%%%%%%%%%%%%%%%%%%%%%%%%%%%%%%%%%%%%%%%%%%%%%%%%%%%%

Sautoy's \textit{The Music of The Primes} \cite{sautoy} tells the story of how Mersenne first began to think about primes. On Christmas in 1640, Pierre de Fermat wrote a note to Mersenne about a subject he was working on. Fermat and Mersenne corresponded frequently, but this note sparked an idea that has permanently written Mersenne in the history of prime numbers. Fermat wrote of his discovery that there are primes that can be written as the sum of two squares.  This struck a chord\footnote{pun intended: Mersenne is known as the father of acoustics} with Mersenne and within a few years he had formulated his now famous equation $2^p - 1$, which we will talk more about later.  

It is interesting to note the relation of that formula to Mersenne's love of music.  Doubling a frequency makes it an octave higher, and therefore powers of 2 are harmonic notes.  It is then convenient to draw a parallel between notes and numbers, where subtracting one from  a note would seem dissonant, a sort of discordant ``prime'' note. 

During Mersenne's life, he contributed to a wide cross-section of fields: natural philosophy, acoustics, music, mechanics, optics, and scientific communication\cite{scibio}. Clearly, Mersenne was one of the more prolific thinkers of his time period, if not in all of history. Perhaps the best way to describe Mersenne's pursuit of all fields of science is to quote from his \emph{Les preludes de lharmonie universelle}:

\begin{quote}
The sciences have sworn among themselves an inviolable partnership; it is almost impossible to separate them, for they would rather suffer than be torn apart; and if anyone persists in doing so, he gets for his trouble only imperfect and confused fragments. \cite{lfrench}
\end{quote}

Born in September of 1588, the young Mersenne entered a Jesuit college for five years, then proceeded to study theology for two years elsewhere. He later joined the little known Minim order of monks and in 1619 entered a convent in Paris, where he stayed until the end of his life.

The Minim order recognized that his greatest use was as an intellectual and he spent his life writing works in a myriad of fields, and corresponding with others throughout Europe on matters of mathematics and more. Mersenne also played an important role in shepherding European science during his lifetime, in an era when scientific thinking was not respected, and potentially dangerous. In 1644, Mersenne published his most famous work, \emph{Cogita Physico-Mathematica}, in which he proposed a formula for generating prime numbers. These became known as Mersenne Primes.

%%%%%%%%%%%%%%%%%%%%%%%%%%%%%%%%%%%%%%%%%%%%%%%%%%%%%%%%%%%%%%%
\subsection{What is a Mersenne Prime?}
%%%%%%%%%%%%%%%%%%%%%%%%%%%%%%%%%%%%%%%%%%%%%%%%%%%%%%%%%%%%%%%

A Mersenne number is a power of two less one, of the form
\begin{align}
\scalebox{1.3}{$M_p = 2^p - 1$}
\label{eqn:mp}
\end{align}
When $M_p$ is prime (as it is for some primes $p$), then $M_p$ is considered a Mersenne Prime. Currently,
there are forty-seven known Mersenne Primes, ranging from $M_2 = 3$ to $M_{43,112,609}$, the massive number mentioned in the first section.

If $2^p-1$ is a prime, then so too will be $p$. And, contrapositively, if $p$ is composite,
so too will be $2^p-1$. \cite{LighNeal}
\begin{thm}
If $2^p-1$ is prime, then $p$ is prime
\end{thm}
\begin{proof}
 If $p$ is composite, it can be written $p = ab$ ($a,b \ne 1$). It is easy to notice, then,
that $2^{ab} - 1 = (2^a)^b - 1$ factors into 
\begin{equation}
2^{ab} - 1 = (2^a-1)(2^{a(b-1)} + 2^{a(b-2)} \ldots 2^{a} + 1)\qedhere
\label{eqn:composite}
\end{equation}
\end{proof}
This is a useful property in reducing the search space for Mersenne Primes. Although Mersenne Prime hunters
(i.e. the GIMPS project from section \ref{sec:gimp}) only search for prime $p$, it is still no easy task to determine the primality of the exponents on the order of $10^{7}$.



