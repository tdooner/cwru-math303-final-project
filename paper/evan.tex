%%%%%%%%%%%%%%%%%%%%%%%%%%%%%%%%%%%%%%%%%%%%%%%%%%%%%%%%%%%%%%%
\section{Background}
%%%%%%%%%%%%%%%%%%%%%%%%%%%%%%%%%%%%%%%%%%%%%%%%%%%%%%%%%%%%%%%

Prime numbers are one of the most important, interesting, and ultimately frustrating subjects in number theory. These integers that have no factors other themselves and 1 are seemingly randomly distributed along the number line and defy all efforts to untangle their mystery. One of the challenges associated with primes, as with any number that has no simple generating function is to find very large ones.  Today, the largest known primes are all Mersenne Primes.  On September 15, 2008 the Great Internet Mersenne Prime Search announced that they had found the largest known prime $2^{43,112,609}-1$, a number that is over 12,000,000 digits long \cite{gimps}.  Mersenne Primes are primes of the form $2^p - 1$ where $p$ is a prime.  These numbers also have quite a few interesting properties that have made them a popular area to work in. There is no sign that Mersenne Primes will lose their dominance of the search for ever larger primes any time soon.  The study of these numbers began some 400 years ago with a French monk named Marin Mersenne. 

%%%%%%%%%%%%%%%%%%%%%%%%%%%%%%%%%%%%%%%%%%%%%%%%%%%%%%%%%%%%%%%
\subsection{}
%%%%%%%%%%%%%%%%%%%%%%%%%%%%%%%%%%%%%%%%%%%%%%%%%%%%%%%%%%%%%%%

Sautoy's \textit{The Music of The Primes} \cite{sautoy} tells the story of how Mersenne first began to think about primes. \textbf{TELL STORY HERE!}  

The Complete Dictionary of Scientific Biography\cite{scibio} lists natural philosophy, acoustics, music, mechanics, optics, and scientific communication as areas that Mersenne made contributions to.  It is clear by this that he was one of the more prolific thinkers of his time period, if not in all of history. Mersenne played an important role in shepherding European science during his lifetime, in an era when scientific thinking was not respected, and potentially dangerous.  Born in September of 1588, the young Mersenne entered a Jesuit college for five years, then proceeded to study theology for two years elsewhere. He later joined the little known Minim order of monks and in 1619 entered a convent in Paris, where he stayed until the end of his life.  

The Minim order recognized that his greatest use was as an intellectual and he spent his life writing works in a myriad of fields, and corresponding with others throughout Europe on matters of mathematics and more. Perhaps the best way to describe Mersenne is to give a quote from him directly. In \emph{Les preludes de lharmonie universelle} he writes \cite{lfrench}

\begin{quote}
The sciences have sworn among themselves an inviolable partnership; it is almost impossible to separate them, for they would rather suffer than be torn apart; and if anyone persists in doing so, he gets for his trouble only imperfect and confused fragments. 
\end{quote}

In 1644, Mersenne published his most famous work, \emph{Cogita Physico-Mathematica}, in which he proposed a formula for generating prime numbers. These became known as Mersenne Primes.

%%%%%%%%%%%%%%%%%%%%%%%%%%%%%%%%%%%%%%%%%%%%%%%%%%%%%%%%%%%%%%%
\subsection{What is a Mersenne Prime?}
%%%%%%%%%%%%%%%%%%%%%%%%%%%%%%%%%%%%%%%%%%%%%%%%%%%%%%%%%%%%%%%

A Mersenne number is a power of two less one, of the form
\begin{align}
M_p = 2^p - 1
\end{align}
The integer p must itself be a prime number in order to generate a Mersenne prime. If p is a
composite number, a composite number will be generated. This can be shown by...
%TODO: ADD PROOF that composite powers will create composite mersennes

%%%%%%%%%%%%%%%%%%%%%%%%%%%%%%%%%%%%%%%%%%%%%%%%%%%%%%%%%%%%%%%
\subsection{The Mersenne Conjecture}
%%%%%%%%%%%%%%%%%%%%%%%%%%%%%%%%%%%%%%%%%%%%%%%%%%%%%%%%%%%%%%%

The first of three Mersenne Conjectures appeared in Mersenne’s 1644 publication of Cogita
Physico-Mathematica, in which Mersenne postulated that primes are generated for p = 2, 3, 5,
7, 13, 17, 19, 31, 67, 127, and 257. As it turns out, Mersenne missed three primes for which his
formula generates a prime, namely 61, 89, and 107. He also missed the fact that the primes 67
and 257 do not generate primes.

The second Mersenne Conjecture was proposed 1989 by P. T. Bateman, J. L. Selfridge and
Wagstaff, Jr., S. S. in the article “The New Mersenne Conjecture.” This conjecture states that if
two of the three following conditions are true, then the third must also hold true~\cite{utm.edu-newconjecture}

\begin{enumerate}
\item $p = 2^k \pm 1$ or $p = 4k \pm 3$ (where $k$ is a natural number)
\item $2^p – 1$ is a prime (more specifically, a Mersenne Prime)
\item $(2^p – 1)/3$ is a prime
\end{enumerate}

As it stands, the New Mersenne Prime Conjecture has yet to be disproven.

The third and most recent Mersenne Conjecture is the Lenstra–Pomerance–Wagstaff
Conjecture. They propose that there are infinitely many primes, and that the number of
Mersenne Primes less than a number x may be approximated by the following formula~\cite{utm.edu-heuristic}

\begin{align}
e^{\gamma} \times \log \left( \dfrac{\log{x}}{\log{2}} \right)
\end{align}

Where the exponential of $\gamma$, called the Euler-Mascheroni constant, is approximately
1.78107.
